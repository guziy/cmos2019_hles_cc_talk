\documentclass{beamer}
\usetheme{metropolis}           % Use metropolis theme
\title{Heavy lake-effect snowfall events for the Laurentian Great Lakes region for current and future climates}
\date{July 2019}
\author{O. Huziy\textsuperscript{1,2,3} and L. Sushama\textsuperscript{2}, L. Leon\textsuperscript{3}, R. Yerubandi\textsuperscript{3}}
\institute{
  \textsuperscript{1}Environement and Climate Change Canada,\\
  \textsuperscript{2}McGill University,\\
  \textsuperscript{3}Université du Québec à Montréal
}

\begin{document}
  \maketitle

  \begin{frame}{Outline}
    \begin{itemize}
      \item Motivation and previous work
      \item Methods
      \begin{itemize}
        \item Models and simulation configurations
        \item Heavy Lake Effect Snowfall (HLES) search algorithm
      \end{itemize}

      \item Results
      \begin{itemize}
        \item HLES in current climate, validation
        \item Projected changes to HLES
      \end{itemize}
    \end{itemize}
  \end{frame}

  \section{Motivation and previous work}
  \begin{frame}{Objectives}
    \begin{itemize}
      \item Apply a more advanced tool to study projected changes to HLES in the Great Lakes region: coupled (only over GL) GEM-NEMO system.
      \item Look into how HLES is going to change during different sub-seasons of the cold season: ND, JF, MA.
      \item Study links between HLES and other near-surface and surface fields: temperature, precipitation, ice concentration in current and future climate.
    \end{itemize}
  \end{frame}

  \section{Models and simulation configurations}
  \begin{frame}{Models and simulation configurations}
    todo
  \end{frame}

  \section{HLES in current climate}
  \begin{frame}{Outline}
    todo
  \end{frame}


  \section{Projected changes to HLES}
  \begin{frame}{Outline}
    todo
  \end{frame}


%% No section here
  \begin{frame}{Conclusions}
    todo
  \end{frame}

  \begin{frame}{References}
    todo
  \end{frame}

  \begin{frame}[standout]
    Questions?
  \end{frame}

\appendix





\end{document}
